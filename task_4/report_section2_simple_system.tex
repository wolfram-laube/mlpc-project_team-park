\section{Simple Speech Command Detection System}
\subsection{System Description}
Our initial system employs a sliding window approach on the audio stream, where overlapping segments are analyzed to detect keywords. The system uses a window size of 1 second with a 0.5-second hop size to ensure thorough coverage of the audio stream.

\subsection{Thresholding and Combining Predictions}
To detect full speech commands, we combine keyword predictions by applying a probability threshold. Detected keywords within a close temporal vicinity are combined to form complete commands. Heuristics such as consecutive detections are considered part of the same command sequence.

\subsection{Cost Minimization Strategies}
To minimize costs, we adjusted the decision thresholds to reduce false positives. Post-processing rules were implemented to filter out unlikely command sequences, such as ensuring that action keywords follow device keywords.

\subsection{Evaluation Setup and Results}
We validated our system using a subset of the annotated scenes from the provided dataset. The evaluation involved a 70/30 training-validation split. The results indicated a reduction in costs compared to the naive baseline, as detailed in Table \ref{tab:simple_system_results}.

\begin{table}[h]
\centering
\begin{tabular}{lccc}
\toprule
Metric & Naive Baseline & Simple System \\
\midrule
True Positives & 60 & 70 \\
False Negatives & 10 & 5 \\
False Positives & 30 & 15 \\
Cross-Triggers & 10 & 8 \\
Total Cost & 5 & -10 \\
\bottomrule
\end{tabular}
\caption{Evaluation results comparing the naive baseline and the simple speech command detection system.}
\label{tab:simple_system_results}
\end{table}
