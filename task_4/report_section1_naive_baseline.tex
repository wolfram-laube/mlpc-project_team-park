\section{Establishing a Naive Baseline}
The naive baseline system for detecting speech commands was a simple keyword spotting algorithm
that identifies any segment of speech as a command without considering context or surrounding words.
This approach uses basic voice activity detection (VAD) to segment the audio and classifies any speech segment containing known keywords as a command.

\subsection{Expected Costs}
The naive baseline incurs high costs due to a significant number of false positives and cross-triggers.
For example, any detected speech might trigger a command, leading to:

\begin{itemize}
  \item \textbf{True Positives (TP):} Correctly identified commands within the correct time frame.
  \item \textbf{False Negatives (FN):} Commands that are present but not detected.
  \item \textbf{False Positives (FP):} Non-command speech segments detected as commands.
  \item \textbf{Cross-Triggers (CT):} Misrecognized commands.
\end{itemize}

Assuming a scenario where the baseline detects 100 events, with 60 being correct commands, 30 being unrelated speech (FP), and 10 missed commands (FN),
the cost can be calculated as follows:

\begin{align*}
\text{TP Cost} &= 60 \times -1 = -60 \\
\text{FN Cost} &= 10 \times 0.5 = 5 \\
\text{FP Cost} &= 30 \times 2 = 60 \quad (\text{assuming an average cost of 2 for FP}) \\
\text{Total Cost} &= -60 + 5 + 60 = 5
\end{align*}
