\documentclass{beamer}
\usetheme{Madrid} % This is a simple and clean theme, but you can change it to other themes like: Berlin, Warsaw, etc.

\title{Speech Command Detection in Noisy Environments}
\subtitle{A Deep Learning Approach to Audio Classification}
\author{Team Park: D. Hörtenhuber, O. König, W. Laube}
\institute{JKU \\ MLPC}
\date{\today}

\begin{document}

\begin{frame}
  \titlepage
\end{frame}

\begin{frame}{Introduction}
  \begin{itemize}
    \item \textbf{Objective:} Develop a robust speech command detection system for smart home environments.
    \item \textbf{Scope:} Detect commands in noisy, continuous audio streams.
    \item \textbf{Goals:}
      \begin{itemize}
        \item Minimize detection costs.
        \item Evaluate performance in realistic scenarios.
      \end{itemize}
  \end{itemize}
\end{frame}

\begin{frame}{System Architecture}
  \begin{itemize}
    \item \textbf{Preprocessing:}
      \begin{itemize}
        \item Voice Activity Detection (VAD)
        \item Feature Extraction (MFCCs)
      \end{itemize}
    \item \textbf{Model:}
      \begin{itemize}
        \item CNN with multiple Conv1D layers
        \item Max pooling and dropout for regularization
      \end{itemize}
    \item \textbf{Post-processing:}
      \begin{itemize}
        \item Sliding window approach
        \item Thresholding and command formation
      \end{itemize}
  \end{itemize}
\end{frame}

\begin{frame}{Hypothesis 1: Hyperparameter Tuning}
  \begin{itemize}
    \item \textbf{Objective:} Improve model performance by tuning hyperparameters.
    \item \textbf{Parameters Tuned:}
      \begin{itemize}
        \item Learning rate
        \item Number of layers
        \item Dropout rate
      \end{itemize}
    \item \textbf{Outcome:}
      \begin{itemize}
        \item Increased accuracy
        \item Reduced false positives
      \end{itemize}
    \item \textbf{Results:}
      \begin{itemize}
        \item Example: Accuracy improved from 85\% to 92\%
      \end{itemize}
  \end{itemize}
\end{frame}

\begin{frame}{Hypothesis 2: Ensembling}
  \begin{itemize}
    \item \textbf{Objective:} Enhance robustness by combining multiple models.
    \item \textbf{Models Used:}
      \begin{itemize}
        \item Multiple CNNs with different architectures
        \item Ensemble voting mechanism
      \end{itemize}
    \item \textbf{Outcome:}
      \begin{itemize}
        \item Reduced false positives and cross-triggers
      \end{itemize}
    \item \textbf{Results:}
      \begin{itemize}
        \item Example: Total cost reduced by 15\%
      \end{itemize}
  \end{itemize}
\end{frame}

\begin{frame}{Hypothesis 3: Data Augmentation}
  \begin{itemize}
    \item \textbf{Objective:} Improve model robustness to noise and variations.
    \item \textbf{Techniques:}
      \begin{itemize}
        \item Adding background noise
        \item Pitch and speed variations
      \end{itemize}
    \item \textbf{Outcome:}
      \begin{itemize}
        \item Improved performance in noisy environments
      \end{itemize}
    \item \textbf{Results:}
      \begin{itemize}
        \item Example: Accuracy increased by 5\% in noisy conditions
      \end{itemize}
  \end{itemize}
\end{frame}

\begin{frame}{Conclusion and Future Work}
  \begin{itemize}
    \item \textbf{Summary:}
      \begin{itemize}
        \item Developed a robust system for speech command detection
        \item Significant improvements through hyperparameter tuning, ensembling, and data augmentation
      \end{itemize}
    \item \textbf{Future Work:}
      \begin{itemize}
        \item Enhance noise robustness
        \item Incorporate user-specific adaptations
        \item Explore real-time processing capabilities
      \end{itemize}
  \end{itemize}
\end{frame}

\end{document}
