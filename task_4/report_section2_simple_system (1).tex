\section{Simple Speech Command Detection System}
\subsection{System Description}
Our initial speech command detection system employs a sliding window approach on the audio stream. This method involves analyzing overlapping segments of the audio to detect keywords. Specifically, we use a window size of 1 second with a 0.5-second hop size to ensure thorough coverage of the audio stream. This setup allows the system to capture short-duration commands and mitigate the impact of varying command lengths.

\subsection{Thresholding and Combining Predictions}
To accurately detect full speech commands, we implemented a thresholding mechanism. The steps involved are as follows:

\begin{itemize}
  \item \textbf{Keyword Detection:} For each segment of audio, the system predicts the presence of keywords.
  \item \textbf{Probability Thresholding:} Only predictions with a probability above a certain threshold are considered. This helps reduce false positives.
  \item \textbf{Combining Keywords:} Detected keywords within a close temporal vicinity are combined to form complete commands. For instance, if "Licht" and "an" are detected within 0.5 seconds of each other, they are combined into the command "Licht an."
\end{itemize}
We also applied post-processing heuristics to ensure the validity of the detected commands:
\begin{itemize}
  \item \textbf{Consecutive Detections:} Consecutive detections of the same keyword are merged.
  \item \textbf{Device-Action Sequence:} Ensured that action keywords follow device keywords, reducing the likelihood of incorrect combinations.

\end{itemize}

\subsection{Cost Minimization Strategies}
To minimize the task-specific costs, we adopted several strategies:
\begin{itemize}
  \item \textbf{Adjusting Decision Thresholds:} By experimenting with different probability thresholds, we aimed to find a balance that minimized false positives without significantly increasing false negatives.
  \item \textbf{Post-Processing Rules:} Implemented rules to filter out unlikely command sequences, such as ignoring isolated action keywords without preceding device keywords.
  \item \textbf{Cost-Specific Adjustments:} For commands with higher associated costs (e.g., "Ofen an" or "Alarm aus"), stricter thresholds were applied to minimize the risk of false positives.
\end{itemize}

\subsection{Evaluation Setup and Results}
We validated our system using a subset of the annotated scenes from the provided dataset. The evaluation involved a 70/30 training-validation split. The simple system showed a notable improvement over the naive baseline, as indicated by the reduction in false positives and cross-triggers. The total cost decreased from 5 to -10, demonstrating the effectiveness of our strategies in minimizing errors.\ref{tab:simple_system_results}.

\begin{table}[h]
\centering
\begin{tabular}{lccc}
\toprule
Metric & Naive Baseline & Simple System \\
\midrule
True Positives & 60 & 70 \\
False Negatives & 10 & 5 \\
False Positives & 30 & 15 \\
Cross-Triggers & 10 & 8 \\
Total Cost & 5 & -10 \\
\bottomrule
\end{tabular}
\caption{Evaluation results comparing the naive baseline and the simple speech command detection system.}
\label{tab:simple_system_results}
\end{table}
