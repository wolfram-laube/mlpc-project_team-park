\documentclass{article}


% if you need to pass options to natbib, use, e.g.:
%     \PassOptionsToPackage{numbers, compress}{natbib}
% before loading neurips_2023

% ready for submission
\usepackage[final]{neurips_2023}

% to avoid loading the natbib package, add option nonatbib:
%    \usepackage[nonatbib]{neurips_2023}


\usepackage[utf8]{inputenc} % allow utf-8 input
\usepackage[T1]{fontenc}    % use 8-bit T1 fonts
\usepackage{hyperref}       % hyperlinks
\usepackage{url}            % simple URL typesetting
\usepackage{booktabs}       % professional-quality tables
\usepackage{amsfonts}       % blackboard math symbols
\usepackage{nicefrac}       % compact symbols for 1/2, etc.
\usepackage{microtype}      % microtypography
\usepackage{xcolor}         % colors


\title{MLPC Report Template 2024}


% The \author macro works with any number of authors. There are two commands
% used to separate the names and addresses of multiple authors: \And and \AND.
%
% Using \And between authors leaves it to LaTeX to determine where to break the
% lines. Using \AND forces a line break at that point. So, if LaTeX puts 3 of 4
% authors names on the first line, and the last on the second line, try using
% \AND instead of \And before the third author name.


\author{% 
  Team TEAMNAME \AND
  Author Name 1
  \And
  Author Name 2 
  \And 
  Author Name 3 
  \And 
  Author Name 4
}


\begin{document}


\maketitle


\begin{contributions}
  Write here briefly and concisely, who of the authors worked on which tasks / questions / etc. \\ \textcolor{gray}{E.g.: Verena Praher did the entire experimental setup, including the data-split and selection of features and pre-processing. Florian Schmid and Paul Primus trained four different classifiers for the task, and Katharina Hoedt was responsible for the evaluation of the classifiers, generating figures and writing the report.}
\end{contributions}


\section{Submission of MLPC reports}

Please read the instructions below carefully and follow them faithfully. Note that this template is based on the official Neurips 2023 template. In your report, you may use three levels of headings, as described in what follows. 

\section{Headings: first level}
\label{sec:headings}

This is a first level heading. 

\subsection{Headings: second level}

This is a second level heading. 


\subsubsection{Headings: third level}

And this is a third level heading. Make sure to structure your report s.t. no deeper levels are necessary. 

\section{Footnotes, Figures and Tables}

\subsection{Footnotes}
Footnotes should be used sparingly. Note that footnotes are properly typeset \emph{after} punctuation marks.\footnote{As in this example.}


\subsection{Figures}


\begin{figure}
  \centering
  \fbox{\rule[-.5cm]{0cm}{4cm} \rule[-.5cm]{4cm}{0cm}}
  \caption{Sample figure caption.}
  \label{fig:example}
\end{figure}


All artwork must be neat, clean, and legible. Lines should be dark enough for
purposes of reproduction. You may use color figures. Please refer to all your figures in text, by using e.g., Figure~\ref{fig:example}. 

\subsection{Tables}
All tables must be centered, neat, clean and legible. Please refer to all your tables in text, by using e.g., Table~\ref{tab:example}.

Note that publication-quality tables \emph{do not contain vertical rules.} We
strongly suggest the use of the \verb+booktabs+ package.\footnote{\url{https://www.ctan.org/pkg/booktabs}}


\begin{table}
  \caption{Sample table title}
  \label{tab:example}
  \centering
  \begin{tabular}{lll}
    \toprule
    \multicolumn{2}{c}{Part}                   \\
    \cmidrule(r){1-2}
    Name     & Description     & Size ($\mu$m) \\
    \midrule
    Dendrite & Input terminal  & $\sim$100     \\
    Axon     & Output terminal & $\sim$10      \\
    Soma     & Cell body       & up to $10^6$  \\
    \bottomrule
  \end{tabular}
\end{table}


\section{Final instructions}

Do not change any aspects of the formatting parameters in the style files. In
particular, do not modify the width or length of the rectangle the text should
fit into, and do not change font sizes (this will result in a deduction of points). 
Please note that pages should be numbered, and adhere to the given \emph{page limit} to avoid further point deductions. Your final submission should be a \texttt{pdf} file.

%%%%%%%%%%%%%%%%%%%%%%%%%%%%%%%%%%%%%%%%%%%%%%%%%%%%%%%%%%%%


\end{document}