\section{Conclusion}

\subsection{Summary of Findings}
In this task, we evaluated two different classifiers—Random Forest and Convolutional Neural Network (CNN)—for the recognition of spoken commands. The analysis involved preprocessing the audio data, implementing the classifiers, and assessing their performance in both ideal and realistic conditions. The key findings are as follows:

\begin{itemize}
    \item \textbf{Preprocessing and Feature Extraction:} Effective preprocessing steps, including normalization and Independent Component Analysis (ICA), were essential in enhancing the quality of the raw audio data. The provided feature set was comprehensive, but the CNN demonstrated strong performance even with minimal feature engineering.

    \item \textbf{Random Forest:} This classifier performed robustly with a high overall accuracy of 91\%. It was particularly effective for most keywords, although it struggled with phonetically similar words like "Ofen" and "offen".

    \item \textbf{CNN:} The CNN outperformed the Random Forest classifier with an accuracy of 92.26\%, demonstrating superior performance in recognizing keywords even in the presence of background noise. However, it still faced challenges with phonetically close keywords and varied speaker accents.

    \item \textbf{Realistic Scene Analysis:} Qualitative evaluation revealed that high background noise and phonetically similar keywords were the primary sources of misclassification. These issues persisted even after applying advanced noise reduction techniques like ICA.
\end{itemize}

\subsection{Future Research Directions}
To further enhance the performance of spoken command recognition systems, the following research directions are recommended:

\begin{itemize}
    \item \textbf{Advanced Preprocessing Techniques:} Investigating more sophisticated noise reduction and signal enhancement techniques could improve the robustness of classifiers in noisy environments.

    \item \textbf{Data Augmentation:} Creating augmented datasets with variations in accents, pronunciations, and phonetically similar keywords can help in making the model more resilient to these variations.

    \item \textbf{Phonetic Feature Integration:} Incorporating phonetic features or using phoneme-based models might help in distinguishing between similar sounding words.

    \item \textbf{Enhanced Model Architectures:} Exploring more advanced architectures, such as attention mechanisms or recurrent neural networks, could provide better contextual understanding and improve recognition accuracy.

    \item \textbf{Real-world Testing:} Conducting extensive testing in diverse real-world environments will be crucial for understanding the practical challenges and iteratively improving the models.

    \item \textbf{User Feedback Integration:} Implementing mechanisms to gather and learn from user feedback could help in continuously refining the model's performance in real-time applications.
\end{itemize}

In conclusion, the CNN classifier demonstrated the best overall performance among the evaluated models, particularly in noisy environments. However, addressing the identified challenges—especially with phonetically similar keywords and diverse speaker accents—will be crucial for developing a robust and reliable spoken command recognition system.
